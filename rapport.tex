\documentclass[10pt]{report}

\usepackage[utf8]{inputenc}
\usepackage[french]{babel}
\usepackage{cancel}
\usepackage{amsmath}
\usepackage{amsfonts}
\usepackage{amssymb}
\usepackage{graphicx}

\begin{document}

\title{Rapport - Devoir 1}
\date{Octobre 2010}
\author{Vincent Foley-Bourgon (FOLV08078309) \and
  Eric Thivierge (THIE09016601)}

\maketitle

\section{Fonctionnement général du programme}

Le programme effectue, dans l'ordre, les opérations suivantes:

\begin{enumerate}
  \item Lecture d'une chaîne.
  \item Détection des erreurs de syntaxe.
  \item Construction d'un arbre syntaxique abstrait.
  \item Évaluation du résultat.
  \item Détection des erreurs de sémantique.
  \item Affichage.
  \item Retour à l'étape 1.
\end{enumerate}


\subsection{Lecture d'une chaîne}

La lecture de la chaîne entrée par l'utilisateur se fait caractère par
caractère.  Cette lecture est entre-mêlée avec la détection d'erreurs
de syntaxe et la construction de l'ASA (décrits ci-dessous).

\subsection{Détection des erreurs de syntaxe}

Aussitôt qu'une erreur de syntaxe est détectée, la fonction qui génère
l'ASA va vider le tampon d'entrée, libérer les structures de données
temporaires et retourner un code d'erreur.

\subsection{Construction d'un arbre syntaxique abstrait}

Lorsqu'un nombre est lu entièrement, il est ajouté à une pile
contenant des expressions (une structure qui sera définie plus tard).
Lorsqu'un opérateur est lu, on combine l'opérateur avec les deux
dernières expressions sur la pile, et on ajoute cette nouvelle
expression sur la pile.

\subsection{Évaluation du résultat}

Lorsqu'un ASA est complet, il est parcouru récursivement à partir de
la racine et à chaque noeud, on applique l'opérateur à ses opérandes.

\subsection{Détection des erreurs de sémantque}

Si durant l'évaluation du résultat une division par zéro est détectée,
un code d'erreur est activé.

\subsection{Affichage}

Si l'expression était valide syntaxiquement et sémentiquement, on
affiche sa représentation dans les syntaxes Scheme, C et Postscript
ainsi que le résultat.  Autrement, un message d'erreur indique la
nature du problème de l'expression.


\section{Représentation des ASAs}

Les arbres de syntaxe abstraite sont des arbres binaires où les noeuds
internes sont des opérateurs et les feuilles sont des nombres.

Dans l'implémentation en C, cette structure est représenté par le type
suivant:

\begin{verbatim}
struct Expr {
	enum { operand, expr } type;
	union {
		Number number;
		struct BinaryOperator {
			enum Operator operator;
			struct Expr *left;
			struct Expr *right;
		} expression;
	} _;
};
\end{verbatim}

L'union stockera soit un nombre ou une expression binaire comportant
un opérateur et deux sous-expressions.  Le champ \emph{type} indique
quel champ de l'union est présentement stocké en mémoire.







\end{document}
